
% Default to the notebook output style

    


% Inherit from the specified cell style.




    
\documentclass[11pt]{article}

    
    
    \usepackage[T1]{fontenc}
    % Nicer default font (+ math font) than Computer Modern for most use cases
    \usepackage{mathpazo}

    % Basic figure setup, for now with no caption control since it's done
    % automatically by Pandoc (which extracts ![](path) syntax from Markdown).
    \usepackage{graphicx}
    % We will generate all images so they have a width \maxwidth. This means
    % that they will get their normal width if they fit onto the page, but
    % are scaled down if they would overflow the margins.
    \makeatletter
    \def\maxwidth{\ifdim\Gin@nat@width>\linewidth\linewidth
    \else\Gin@nat@width\fi}
    \makeatother
    \let\Oldincludegraphics\includegraphics
    % Set max figure width to be 80% of text width, for now hardcoded.
    \renewcommand{\includegraphics}[1]{\Oldincludegraphics[width=.8\maxwidth]{#1}}
    % Ensure that by default, figures have no caption (until we provide a
    % proper Figure object with a Caption API and a way to capture that
    % in the conversion process - todo).
    \usepackage{caption}
    \DeclareCaptionLabelFormat{nolabel}{}
    \captionsetup{labelformat=nolabel}

    \usepackage{adjustbox} % Used to constrain images to a maximum size 
    \usepackage{xcolor} % Allow colors to be defined
    \usepackage{enumerate} % Needed for markdown enumerations to work
    \usepackage{geometry} % Used to adjust the document margins
    \usepackage{amsmath} % Equations
    \usepackage{amssymb} % Equations
    \usepackage{textcomp} % defines textquotesingle
    % Hack from http://tex.stackexchange.com/a/47451/13684:
    \AtBeginDocument{%
        \def\PYZsq{\textquotesingle}% Upright quotes in Pygmentized code
    }
    \usepackage{upquote} % Upright quotes for verbatim code
    \usepackage{eurosym} % defines \euro
    \usepackage[mathletters]{ucs} % Extended unicode (utf-8) support
    \usepackage[utf8x]{inputenc} % Allow utf-8 characters in the tex document
    \usepackage{fancyvrb} % verbatim replacement that allows latex
    \usepackage{grffile} % extends the file name processing of package graphics 
                         % to support a larger range 
    % The hyperref package gives us a pdf with properly built
    % internal navigation ('pdf bookmarks' for the table of contents,
    % internal cross-reference links, web links for URLs, etc.)
    \usepackage{hyperref}
    \usepackage{longtable} % longtable support required by pandoc >1.10
    \usepackage{booktabs}  % table support for pandoc > 1.12.2
    \usepackage[inline]{enumitem} % IRkernel/repr support (it uses the enumerate* environment)
    \usepackage[normalem]{ulem} % ulem is needed to support strikethroughs (\sout)
                                % normalem makes italics be italics, not underlines
    

    
    
    % Colors for the hyperref package
    \definecolor{urlcolor}{rgb}{0,.145,.698}
    \definecolor{linkcolor}{rgb}{.71,0.21,0.01}
    \definecolor{citecolor}{rgb}{.12,.54,.11}

    % ANSI colors
    \definecolor{ansi-black}{HTML}{3E424D}
    \definecolor{ansi-black-intense}{HTML}{282C36}
    \definecolor{ansi-red}{HTML}{E75C58}
    \definecolor{ansi-red-intense}{HTML}{B22B31}
    \definecolor{ansi-green}{HTML}{00A250}
    \definecolor{ansi-green-intense}{HTML}{007427}
    \definecolor{ansi-yellow}{HTML}{DDB62B}
    \definecolor{ansi-yellow-intense}{HTML}{B27D12}
    \definecolor{ansi-blue}{HTML}{208FFB}
    \definecolor{ansi-blue-intense}{HTML}{0065CA}
    \definecolor{ansi-magenta}{HTML}{D160C4}
    \definecolor{ansi-magenta-intense}{HTML}{A03196}
    \definecolor{ansi-cyan}{HTML}{60C6C8}
    \definecolor{ansi-cyan-intense}{HTML}{258F8F}
    \definecolor{ansi-white}{HTML}{C5C1B4}
    \definecolor{ansi-white-intense}{HTML}{A1A6B2}

    % commands and environments needed by pandoc snippets
    % extracted from the output of `pandoc -s`
    \providecommand{\tightlist}{%
      \setlength{\itemsep}{0pt}\setlength{\parskip}{0pt}}
    \DefineVerbatimEnvironment{Highlighting}{Verbatim}{commandchars=\\\{\}}
    % Add ',fontsize=\small' for more characters per line
    \newenvironment{Shaded}{}{}
    \newcommand{\KeywordTok}[1]{\textcolor[rgb]{0.00,0.44,0.13}{\textbf{{#1}}}}
    \newcommand{\DataTypeTok}[1]{\textcolor[rgb]{0.56,0.13,0.00}{{#1}}}
    \newcommand{\DecValTok}[1]{\textcolor[rgb]{0.25,0.63,0.44}{{#1}}}
    \newcommand{\BaseNTok}[1]{\textcolor[rgb]{0.25,0.63,0.44}{{#1}}}
    \newcommand{\FloatTok}[1]{\textcolor[rgb]{0.25,0.63,0.44}{{#1}}}
    \newcommand{\CharTok}[1]{\textcolor[rgb]{0.25,0.44,0.63}{{#1}}}
    \newcommand{\StringTok}[1]{\textcolor[rgb]{0.25,0.44,0.63}{{#1}}}
    \newcommand{\CommentTok}[1]{\textcolor[rgb]{0.38,0.63,0.69}{\textit{{#1}}}}
    \newcommand{\OtherTok}[1]{\textcolor[rgb]{0.00,0.44,0.13}{{#1}}}
    \newcommand{\AlertTok}[1]{\textcolor[rgb]{1.00,0.00,0.00}{\textbf{{#1}}}}
    \newcommand{\FunctionTok}[1]{\textcolor[rgb]{0.02,0.16,0.49}{{#1}}}
    \newcommand{\RegionMarkerTok}[1]{{#1}}
    \newcommand{\ErrorTok}[1]{\textcolor[rgb]{1.00,0.00,0.00}{\textbf{{#1}}}}
    \newcommand{\NormalTok}[1]{{#1}}
    
    % Additional commands for more recent versions of Pandoc
    \newcommand{\ConstantTok}[1]{\textcolor[rgb]{0.53,0.00,0.00}{{#1}}}
    \newcommand{\SpecialCharTok}[1]{\textcolor[rgb]{0.25,0.44,0.63}{{#1}}}
    \newcommand{\VerbatimStringTok}[1]{\textcolor[rgb]{0.25,0.44,0.63}{{#1}}}
    \newcommand{\SpecialStringTok}[1]{\textcolor[rgb]{0.73,0.40,0.53}{{#1}}}
    \newcommand{\ImportTok}[1]{{#1}}
    \newcommand{\DocumentationTok}[1]{\textcolor[rgb]{0.73,0.13,0.13}{\textit{{#1}}}}
    \newcommand{\AnnotationTok}[1]{\textcolor[rgb]{0.38,0.63,0.69}{\textbf{\textit{{#1}}}}}
    \newcommand{\CommentVarTok}[1]{\textcolor[rgb]{0.38,0.63,0.69}{\textbf{\textit{{#1}}}}}
    \newcommand{\VariableTok}[1]{\textcolor[rgb]{0.10,0.09,0.49}{{#1}}}
    \newcommand{\ControlFlowTok}[1]{\textcolor[rgb]{0.00,0.44,0.13}{\textbf{{#1}}}}
    \newcommand{\OperatorTok}[1]{\textcolor[rgb]{0.40,0.40,0.40}{{#1}}}
    \newcommand{\BuiltInTok}[1]{{#1}}
    \newcommand{\ExtensionTok}[1]{{#1}}
    \newcommand{\PreprocessorTok}[1]{\textcolor[rgb]{0.74,0.48,0.00}{{#1}}}
    \newcommand{\AttributeTok}[1]{\textcolor[rgb]{0.49,0.56,0.16}{{#1}}}
    \newcommand{\InformationTok}[1]{\textcolor[rgb]{0.38,0.63,0.69}{\textbf{\textit{{#1}}}}}
    \newcommand{\WarningTok}[1]{\textcolor[rgb]{0.38,0.63,0.69}{\textbf{\textit{{#1}}}}}
    
    
    % Define a nice break command that doesn't care if a line doesn't already
    % exist.
    \def\br{\hspace*{\fill} \\* }
    % Math Jax compatability definitions
    \def\gt{>}
    \def\lt{<}
    % Document parameters
    \title{CloudSat Radar and MODIS Cloud Top Height Comparisons for 2015 El-Ni\~{n}o}
    \author{Drew Camron and Matt Cann}
    
    
    

    % Pygments definitions
    
\makeatletter
\def\PY@reset{\let\PY@it=\relax \let\PY@bf=\relax%
    \let\PY@ul=\relax \let\PY@tc=\relax%
    \let\PY@bc=\relax \let\PY@ff=\relax}
\def\PY@tok#1{\csname PY@tok@#1\endcsname}
\def\PY@toks#1+{\ifx\relax#1\empty\else%
    \PY@tok{#1}\expandafter\PY@toks\fi}
\def\PY@do#1{\PY@bc{\PY@tc{\PY@ul{%
    \PY@it{\PY@bf{\PY@ff{#1}}}}}}}
\def\PY#1#2{\PY@reset\PY@toks#1+\relax+\PY@do{#2}}

\expandafter\def\csname PY@tok@w\endcsname{\def\PY@tc##1{\textcolor[rgb]{0.73,0.73,0.73}{##1}}}
\expandafter\def\csname PY@tok@c\endcsname{\let\PY@it=\textit\def\PY@tc##1{\textcolor[rgb]{0.25,0.50,0.50}{##1}}}
\expandafter\def\csname PY@tok@cp\endcsname{\def\PY@tc##1{\textcolor[rgb]{0.74,0.48,0.00}{##1}}}
\expandafter\def\csname PY@tok@k\endcsname{\let\PY@bf=\textbf\def\PY@tc##1{\textcolor[rgb]{0.00,0.50,0.00}{##1}}}
\expandafter\def\csname PY@tok@kp\endcsname{\def\PY@tc##1{\textcolor[rgb]{0.00,0.50,0.00}{##1}}}
\expandafter\def\csname PY@tok@kt\endcsname{\def\PY@tc##1{\textcolor[rgb]{0.69,0.00,0.25}{##1}}}
\expandafter\def\csname PY@tok@o\endcsname{\def\PY@tc##1{\textcolor[rgb]{0.40,0.40,0.40}{##1}}}
\expandafter\def\csname PY@tok@ow\endcsname{\let\PY@bf=\textbf\def\PY@tc##1{\textcolor[rgb]{0.67,0.13,1.00}{##1}}}
\expandafter\def\csname PY@tok@nb\endcsname{\def\PY@tc##1{\textcolor[rgb]{0.00,0.50,0.00}{##1}}}
\expandafter\def\csname PY@tok@nf\endcsname{\def\PY@tc##1{\textcolor[rgb]{0.00,0.00,1.00}{##1}}}
\expandafter\def\csname PY@tok@nc\endcsname{\let\PY@bf=\textbf\def\PY@tc##1{\textcolor[rgb]{0.00,0.00,1.00}{##1}}}
\expandafter\def\csname PY@tok@nn\endcsname{\let\PY@bf=\textbf\def\PY@tc##1{\textcolor[rgb]{0.00,0.00,1.00}{##1}}}
\expandafter\def\csname PY@tok@ne\endcsname{\let\PY@bf=\textbf\def\PY@tc##1{\textcolor[rgb]{0.82,0.25,0.23}{##1}}}
\expandafter\def\csname PY@tok@nv\endcsname{\def\PY@tc##1{\textcolor[rgb]{0.10,0.09,0.49}{##1}}}
\expandafter\def\csname PY@tok@no\endcsname{\def\PY@tc##1{\textcolor[rgb]{0.53,0.00,0.00}{##1}}}
\expandafter\def\csname PY@tok@nl\endcsname{\def\PY@tc##1{\textcolor[rgb]{0.63,0.63,0.00}{##1}}}
\expandafter\def\csname PY@tok@ni\endcsname{\let\PY@bf=\textbf\def\PY@tc##1{\textcolor[rgb]{0.60,0.60,0.60}{##1}}}
\expandafter\def\csname PY@tok@na\endcsname{\def\PY@tc##1{\textcolor[rgb]{0.49,0.56,0.16}{##1}}}
\expandafter\def\csname PY@tok@nt\endcsname{\let\PY@bf=\textbf\def\PY@tc##1{\textcolor[rgb]{0.00,0.50,0.00}{##1}}}
\expandafter\def\csname PY@tok@nd\endcsname{\def\PY@tc##1{\textcolor[rgb]{0.67,0.13,1.00}{##1}}}
\expandafter\def\csname PY@tok@s\endcsname{\def\PY@tc##1{\textcolor[rgb]{0.73,0.13,0.13}{##1}}}
\expandafter\def\csname PY@tok@sd\endcsname{\let\PY@it=\textit\def\PY@tc##1{\textcolor[rgb]{0.73,0.13,0.13}{##1}}}
\expandafter\def\csname PY@tok@si\endcsname{\let\PY@bf=\textbf\def\PY@tc##1{\textcolor[rgb]{0.73,0.40,0.53}{##1}}}
\expandafter\def\csname PY@tok@se\endcsname{\let\PY@bf=\textbf\def\PY@tc##1{\textcolor[rgb]{0.73,0.40,0.13}{##1}}}
\expandafter\def\csname PY@tok@sr\endcsname{\def\PY@tc##1{\textcolor[rgb]{0.73,0.40,0.53}{##1}}}
\expandafter\def\csname PY@tok@ss\endcsname{\def\PY@tc##1{\textcolor[rgb]{0.10,0.09,0.49}{##1}}}
\expandafter\def\csname PY@tok@sx\endcsname{\def\PY@tc##1{\textcolor[rgb]{0.00,0.50,0.00}{##1}}}
\expandafter\def\csname PY@tok@m\endcsname{\def\PY@tc##1{\textcolor[rgb]{0.40,0.40,0.40}{##1}}}
\expandafter\def\csname PY@tok@gh\endcsname{\let\PY@bf=\textbf\def\PY@tc##1{\textcolor[rgb]{0.00,0.00,0.50}{##1}}}
\expandafter\def\csname PY@tok@gu\endcsname{\let\PY@bf=\textbf\def\PY@tc##1{\textcolor[rgb]{0.50,0.00,0.50}{##1}}}
\expandafter\def\csname PY@tok@gd\endcsname{\def\PY@tc##1{\textcolor[rgb]{0.63,0.00,0.00}{##1}}}
\expandafter\def\csname PY@tok@gi\endcsname{\def\PY@tc##1{\textcolor[rgb]{0.00,0.63,0.00}{##1}}}
\expandafter\def\csname PY@tok@gr\endcsname{\def\PY@tc##1{\textcolor[rgb]{1.00,0.00,0.00}{##1}}}
\expandafter\def\csname PY@tok@ge\endcsname{\let\PY@it=\textit}
\expandafter\def\csname PY@tok@gs\endcsname{\let\PY@bf=\textbf}
\expandafter\def\csname PY@tok@gp\endcsname{\let\PY@bf=\textbf\def\PY@tc##1{\textcolor[rgb]{0.00,0.00,0.50}{##1}}}
\expandafter\def\csname PY@tok@go\endcsname{\def\PY@tc##1{\textcolor[rgb]{0.53,0.53,0.53}{##1}}}
\expandafter\def\csname PY@tok@gt\endcsname{\def\PY@tc##1{\textcolor[rgb]{0.00,0.27,0.87}{##1}}}
\expandafter\def\csname PY@tok@err\endcsname{\def\PY@bc##1{\setlength{\fboxsep}{0pt}\fcolorbox[rgb]{1.00,0.00,0.00}{1,1,1}{\strut ##1}}}
\expandafter\def\csname PY@tok@kc\endcsname{\let\PY@bf=\textbf\def\PY@tc##1{\textcolor[rgb]{0.00,0.50,0.00}{##1}}}
\expandafter\def\csname PY@tok@kd\endcsname{\let\PY@bf=\textbf\def\PY@tc##1{\textcolor[rgb]{0.00,0.50,0.00}{##1}}}
\expandafter\def\csname PY@tok@kn\endcsname{\let\PY@bf=\textbf\def\PY@tc##1{\textcolor[rgb]{0.00,0.50,0.00}{##1}}}
\expandafter\def\csname PY@tok@kr\endcsname{\let\PY@bf=\textbf\def\PY@tc##1{\textcolor[rgb]{0.00,0.50,0.00}{##1}}}
\expandafter\def\csname PY@tok@bp\endcsname{\def\PY@tc##1{\textcolor[rgb]{0.00,0.50,0.00}{##1}}}
\expandafter\def\csname PY@tok@fm\endcsname{\def\PY@tc##1{\textcolor[rgb]{0.00,0.00,1.00}{##1}}}
\expandafter\def\csname PY@tok@vc\endcsname{\def\PY@tc##1{\textcolor[rgb]{0.10,0.09,0.49}{##1}}}
\expandafter\def\csname PY@tok@vg\endcsname{\def\PY@tc##1{\textcolor[rgb]{0.10,0.09,0.49}{##1}}}
\expandafter\def\csname PY@tok@vi\endcsname{\def\PY@tc##1{\textcolor[rgb]{0.10,0.09,0.49}{##1}}}
\expandafter\def\csname PY@tok@vm\endcsname{\def\PY@tc##1{\textcolor[rgb]{0.10,0.09,0.49}{##1}}}
\expandafter\def\csname PY@tok@sa\endcsname{\def\PY@tc##1{\textcolor[rgb]{0.73,0.13,0.13}{##1}}}
\expandafter\def\csname PY@tok@sb\endcsname{\def\PY@tc##1{\textcolor[rgb]{0.73,0.13,0.13}{##1}}}
\expandafter\def\csname PY@tok@sc\endcsname{\def\PY@tc##1{\textcolor[rgb]{0.73,0.13,0.13}{##1}}}
\expandafter\def\csname PY@tok@dl\endcsname{\def\PY@tc##1{\textcolor[rgb]{0.73,0.13,0.13}{##1}}}
\expandafter\def\csname PY@tok@s2\endcsname{\def\PY@tc##1{\textcolor[rgb]{0.73,0.13,0.13}{##1}}}
\expandafter\def\csname PY@tok@sh\endcsname{\def\PY@tc##1{\textcolor[rgb]{0.73,0.13,0.13}{##1}}}
\expandafter\def\csname PY@tok@s1\endcsname{\def\PY@tc##1{\textcolor[rgb]{0.73,0.13,0.13}{##1}}}
\expandafter\def\csname PY@tok@mb\endcsname{\def\PY@tc##1{\textcolor[rgb]{0.40,0.40,0.40}{##1}}}
\expandafter\def\csname PY@tok@mf\endcsname{\def\PY@tc##1{\textcolor[rgb]{0.40,0.40,0.40}{##1}}}
\expandafter\def\csname PY@tok@mh\endcsname{\def\PY@tc##1{\textcolor[rgb]{0.40,0.40,0.40}{##1}}}
\expandafter\def\csname PY@tok@mi\endcsname{\def\PY@tc##1{\textcolor[rgb]{0.40,0.40,0.40}{##1}}}
\expandafter\def\csname PY@tok@il\endcsname{\def\PY@tc##1{\textcolor[rgb]{0.40,0.40,0.40}{##1}}}
\expandafter\def\csname PY@tok@mo\endcsname{\def\PY@tc##1{\textcolor[rgb]{0.40,0.40,0.40}{##1}}}
\expandafter\def\csname PY@tok@ch\endcsname{\let\PY@it=\textit\def\PY@tc##1{\textcolor[rgb]{0.25,0.50,0.50}{##1}}}
\expandafter\def\csname PY@tok@cm\endcsname{\let\PY@it=\textit\def\PY@tc##1{\textcolor[rgb]{0.25,0.50,0.50}{##1}}}
\expandafter\def\csname PY@tok@cpf\endcsname{\let\PY@it=\textit\def\PY@tc##1{\textcolor[rgb]{0.25,0.50,0.50}{##1}}}
\expandafter\def\csname PY@tok@c1\endcsname{\let\PY@it=\textit\def\PY@tc##1{\textcolor[rgb]{0.25,0.50,0.50}{##1}}}
\expandafter\def\csname PY@tok@cs\endcsname{\let\PY@it=\textit\def\PY@tc##1{\textcolor[rgb]{0.25,0.50,0.50}{##1}}}

\def\PYZbs{\char`\\}
\def\PYZus{\char`\_}
\def\PYZob{\char`\{}
\def\PYZcb{\char`\}}
\def\PYZca{\char`\^}
\def\PYZam{\char`\&}
\def\PYZlt{\char`\<}
\def\PYZgt{\char`\>}
\def\PYZsh{\char`\#}
\def\PYZpc{\char`\%}
\def\PYZdl{\char`\$}
\def\PYZhy{\char`\-}
\def\PYZsq{\char`\'}
\def\PYZdq{\char`\"}
\def\PYZti{\char`\~}
% for compatibility with earlier versions
\def\PYZat{@}
\def\PYZlb{[}
\def\PYZrb{]}
\makeatother


    % Exact colors from NB
    \definecolor{incolor}{rgb}{0.0, 0.0, 0.5}
    \definecolor{outcolor}{rgb}{0.545, 0.0, 0.0}



    
    % Prevent overflowing lines due to hard-to-break entities
    \sloppy 
    % Setup hyperref package
    \hypersetup{
      breaklinks=true,  % so long urls are correctly broken across lines
      colorlinks=true,
      urlcolor=urlcolor,
      linkcolor=linkcolor,
      citecolor=citecolor,
      }
    % Slightly bigger margins than the latex defaults
    
    \geometry{verbose,tmargin=1in,bmargin=1in,lmargin=1in,rmargin=1in}
    
    

    \begin{document}
    
    
    \maketitle
    


    \section{Introduction}\label{introduction}

The difference in passive and active measurements can be substantial
(Mahesh et al. 2004; Weisz et al. 2007). These differences can be
important when studying sensitive thresholds such as cloud top height.
Accurate cloud top heights are needed for weather and climate models and
can have a large impact on the radiative budget.

This study will be concerned with cloud top height measurements during
the 2015-2016 El Niño in the equatorial pacific. The eastward shift and
enhanced convection associated with El Niño conditions is thought to
produce larger cells of convection, which increases the global average
cloud top height (Xu et al. 2007; Davies and Molloy 2012). Higher
average cloud top heights emit less outgoing longwave radiation and
therefore have a warming effect on the planet. El Niño conditions are
correlated with warmer global average surface temperatures, and accurate
cloud top height retrievals may help us understand a possible
connection. Feeding higher accuracy cloud top heights into weather
models may also produce more accurate forecasts. Comparisons between
active (MODIS) and passive (CloudSat) measurements will be drawn in this
study for convection occurring on the equatorial pacific on 25 December
2015.

The Moderate Resolution Imaging Spectroradiometer (MODIS) instrument on
Aqua has 36 spectral bands with wavelengths ranging from 0.4 to 14.4 µm.
MODIS is an active measuring system with a ±55° scanning window
producing swaths of 2330 km and most products at 1- or 5-km spatial
resolution (King et al. 1992). For this study, we used IR bands 31 and
33-36 with bandwidths ranging from 10.8 to 14.4 µm, which combine to
create the level 2 cloud top height dataset. The dataset has 5-km
spatial resolution.

CloudSat is a 94-GHz nadir-looking cloud radar that tracks 176 seconds
behind Aqua. CloudSat CPR measure backscattered energy from clouds and
precipitation in a 1.5-km across track and 2.5-km along track footprint
and \textasciitilde{}240-m, 125 bin vertical bin resolution for a total
window of 30 km (Marchand et al. 2008). For this study, we used the
level 2 GEOPROF product which provides significant radar reflectivity as
opposed to noise and clutter.

    \section{Code, Analysis, and
Figures}\label{code-analysis-and-figures}

    \subsection{Important Reading}\label{important-reading}

\href{http://fhs.github.io/python-hdf4/}{python-hdf4 (aka pyhdf) module
docs}\\
\href{http://hdfeos.org/zoo/LAADS/MOD08_D3_Cloud_Fraction_Liquid.py}{sample
modis import}\\
\href{http://hdfeos.org/zoo/OTHER/2010128055614_21420_CS_2B-GEOPROF_GRANULE_P_R04_E03.hdf.py}{sample
cloudsat import}

    \subsection{Importing and cleaning
data}\label{importing-and-cleaning-data}

\textbf{Note}: some of the output used to investigate the HDF files has
been squashed to save page space, but the code is still valid.

    \begin{Verbatim}[commandchars=\\\{\}]
{\color{incolor}In [{\color{incolor}1}]:} \PY{o}{\PYZpc{}}\PY{k}{matplotlib} inline
        \PY{k+kn}{from} \PY{n+nn}{pyhdf}\PY{n+nn}{.}\PY{n+nn}{SD} \PY{k}{import} \PY{o}{*}
        \PY{k+kn}{from} \PY{n+nn}{pyhdf}\PY{n+nn}{.}\PY{n+nn}{VS} \PY{k}{import} \PY{o}{*}
        \PY{k+kn}{from} \PY{n+nn}{pyhdf}\PY{n+nn}{.}\PY{n+nn}{V} \PY{k}{import} \PY{o}{*}
        \PY{k+kn}{from} \PY{n+nn}{pyhdf}\PY{n+nn}{.}\PY{n+nn}{HDF} \PY{k}{import} \PY{o}{*}
        \PY{k+kn}{import} \PY{n+nn}{numpy} \PY{k}{as} \PY{n+nn}{np}
        
        \PY{n}{in\PYZus{}modis} \PY{o}{=} \PY{l+s+s1}{\PYZsq{}}\PY{l+s+s1}{Data/aqua\PYZhy{}modis\PYZus{}ctpres\PYZus{}20151225\PYZus{}0140.hdf}\PY{l+s+s1}{\PYZsq{}}
        \PY{n}{in\PYZus{}cs} \PY{o}{=} \PY{l+s+s1}{\PYZsq{}}\PY{l+s+s1}{Data/cloudsat\PYZus{}radar\PYZus{}20151225.hdf}\PY{l+s+s1}{\PYZsq{}}
        
        \PY{n}{m} \PY{o}{=} \PY{n}{SD}\PY{p}{(}\PY{n}{in\PYZus{}modis}\PY{p}{,} \PY{n}{SDC}\PY{o}{.}\PY{n}{READ}\PY{p}{)}
        \PY{n}{c} \PY{o}{=} \PY{n}{SD}\PY{p}{(}\PY{n}{in\PYZus{}cs}\PY{p}{,} \PY{n}{SDC}\PY{o}{.}\PY{n}{READ}\PY{p}{)}
        
        \PY{n}{hm} \PY{o}{=} \PY{n}{HDF}\PY{p}{(}\PY{n}{in\PYZus{}modis}\PY{p}{)}
        \PY{n}{hc} \PY{o}{=} \PY{n}{HDF}\PY{p}{(}\PY{n}{in\PYZus{}cs}\PY{p}{)}
\end{Verbatim}

    These data files are structured differently. The MODIS data stores all
of its coordinates and data in datasets, to be viewed and extracted
below.

    \begin{Verbatim}[commandchars=\\\{\}]
{\color{incolor}In [{\color{incolor}2}]:} \PY{c+c1}{\PYZsh{}m.datasets() \PYZsh{}long list of data}
\end{Verbatim}

    \begin{Verbatim}[commandchars=\\\{\}]
{\color{incolor}In [{\color{incolor}3}]:} \PY{n}{dslat\PYZus{}m} \PY{o}{=} \PY{n}{m}\PY{o}{.}\PY{n}{select}\PY{p}{(}\PY{l+s+s1}{\PYZsq{}}\PY{l+s+s1}{Latitude}\PY{l+s+s1}{\PYZsq{}}\PY{p}{)}
        \PY{n}{dslon\PYZus{}m} \PY{o}{=} \PY{n}{m}\PY{o}{.}\PY{n}{select}\PY{p}{(}\PY{l+s+s1}{\PYZsq{}}\PY{l+s+s1}{Longitude}\PY{l+s+s1}{\PYZsq{}}\PY{p}{)}
        \PY{n}{dsct\PYZus{}m} \PY{o}{=} \PY{n}{m}\PY{o}{.}\PY{n}{select}\PY{p}{(}\PY{l+s+s1}{\PYZsq{}}\PY{l+s+s1}{Cloud\PYZus{}Top\PYZus{}Height}\PY{l+s+s1}{\PYZsq{}}\PY{p}{)}
        
        \PY{n}{lat\PYZus{}m} \PY{o}{=} \PY{n}{dslat\PYZus{}m}\PY{o}{.}\PY{n}{get}\PY{p}{(}\PY{p}{)}    \PY{c+c1}{\PYZsh{} MODIS swath latitude}
        \PY{n}{lon\PYZus{}m} \PY{o}{=} \PY{n}{dslon\PYZus{}m}\PY{o}{.}\PY{n}{get}\PY{p}{(}\PY{p}{)}    \PY{c+c1}{\PYZsh{} MODIS swath longitude}
        \PY{c+c1}{\PYZsh{} converting to [0,360] deg E longitude}
        \PY{n}{lon\PYZus{}m}\PY{p}{[}\PY{n}{lon\PYZus{}m} \PY{o}{\PYZlt{}} \PY{l+m+mi}{0}\PY{p}{]} \PY{o}{=} \PY{n}{lon\PYZus{}m}\PY{p}{[}\PY{n}{lon\PYZus{}m} \PY{o}{\PYZlt{}} \PY{l+m+mi}{0}\PY{p}{]} \PY{o}{+} \PY{l+m+mi}{360}
        \PY{n}{ct\PYZus{}mi} \PY{o}{=} \PY{n}{dsct\PYZus{}m}\PY{o}{.}\PY{n}{get}\PY{p}{(}\PY{p}{)}
        \PY{n}{at} \PY{o}{=} \PY{n}{dsct\PYZus{}m}\PY{o}{.}\PY{n}{attributes}\PY{p}{(}\PY{p}{)}
\end{Verbatim}

    \begin{Verbatim}[commandchars=\\\{\}]
{\color{incolor}In [{\color{incolor}4}]:} \PY{c+c1}{\PYZsh{}at}
\end{Verbatim}

    Here we can pull in some of the provided attributes. The data are not
scaled or offset, but we will make sure to NaN invalid data.

    \begin{Verbatim}[commandchars=\\\{\}]
{\color{incolor}In [{\color{incolor}5}]:} \PY{n}{\PYZus{}FillValue} \PY{o}{=} \PY{n}{at}\PY{p}{[}\PY{l+s+s1}{\PYZsq{}}\PY{l+s+s1}{\PYZus{}FillValue}\PY{l+s+s1}{\PYZsq{}}\PY{p}{]}
        \PY{n}{vra} \PY{o}{=} \PY{n}{at}\PY{p}{[}\PY{l+s+s1}{\PYZsq{}}\PY{l+s+s1}{valid\PYZus{}range}\PY{l+s+s1}{\PYZsq{}}\PY{p}{]}
        \PY{n}{ctname\PYZus{}m} \PY{o}{=} \PY{n}{at}\PY{p}{[}\PY{l+s+s1}{\PYZsq{}}\PY{l+s+s1}{long\PYZus{}name}\PY{l+s+s1}{\PYZsq{}}\PY{p}{]}
        \PY{n}{ctunits\PYZus{}m} \PY{o}{=} \PY{n}{at}\PY{p}{[}\PY{l+s+s1}{\PYZsq{}}\PY{l+s+s1}{units}\PY{l+s+s1}{\PYZsq{}}\PY{p}{]}
        
        \PY{n}{invalid} \PY{o}{=} \PY{n}{np}\PY{o}{.}\PY{n}{logical\PYZus{}or}\PY{p}{(}\PY{n}{ct\PYZus{}mi} \PY{o}{\PYZlt{}} \PY{n}{vra}\PY{p}{[}\PY{l+m+mi}{0}\PY{p}{]}\PY{p}{,} \PY{n}{ct\PYZus{}mi} \PY{o}{\PYZgt{}} \PY{n}{vra}\PY{p}{[}\PY{l+m+mi}{1}\PY{p}{]}\PY{p}{,} \PY{n}{ct\PYZus{}mi} \PY{o}{==} \PY{n}{\PYZus{}FillValue}\PY{p}{)}
        \PY{n}{ct\PYZus{}m} \PY{o}{=} \PY{n}{ct\PYZus{}mi}\PY{o}{.}\PY{n}{astype}\PY{p}{(}\PY{n+nb}{float}\PY{p}{)}    \PY{c+c1}{\PYZsh{} cleaned MODIS cloud\PYZhy{}top height data}
        \PY{n}{ct\PYZus{}m}\PY{p}{[}\PY{n}{invalid}\PY{p}{]} \PY{o}{=} \PY{n}{np}\PY{o}{.}\PY{n}{nan}
\end{Verbatim}

    Now, the cloudsat data have coordinating information hidden in the HDF
vdata, which we will investigate and extract below.

    \begin{Verbatim}[commandchars=\\\{\}]
{\color{incolor}In [{\color{incolor}6}]:} \PY{n}{vsc} \PY{o}{=} \PY{n}{hc}\PY{o}{.}\PY{n}{vstart}\PY{p}{(}\PY{p}{)}
\end{Verbatim}

    \begin{Verbatim}[commandchars=\\\{\}]
{\color{incolor}In [{\color{incolor}7}]:} \PY{c+c1}{\PYZsh{}vsc.vdatainfo()}
\end{Verbatim}

    \begin{Verbatim}[commandchars=\\\{\}]
{\color{incolor}In [{\color{incolor}8}]:} \PY{n}{latc} \PY{o}{=} \PY{n}{vsc}\PY{o}{.}\PY{n}{attach}\PY{p}{(}\PY{l+s+s1}{\PYZsq{}}\PY{l+s+s1}{Latitude}\PY{l+s+s1}{\PYZsq{}}\PY{p}{)}
        \PY{n}{latc}\PY{o}{.}\PY{n}{setfields}\PY{p}{(}\PY{l+s+s1}{\PYZsq{}}\PY{l+s+s1}{Latitude}\PY{l+s+s1}{\PYZsq{}}\PY{p}{)}
        \PY{n}{n}\PY{p}{,} \PY{n}{\PYZus{}}\PY{p}{,} \PY{n}{\PYZus{}}\PY{p}{,} \PY{n}{\PYZus{}}\PY{p}{,} \PY{n}{\PYZus{}} \PY{o}{=} \PY{n}{latc}\PY{o}{.}\PY{n}{inquire}\PY{p}{(}\PY{p}{)}
        \PY{n}{lat\PYZus{}c} \PY{o}{=} \PY{n}{np}\PY{o}{.}\PY{n}{array}\PY{p}{(}\PY{n}{latc}\PY{o}{.}\PY{n}{read}\PY{p}{(}\PY{n}{n}\PY{p}{)}\PY{p}{)}    \PY{c+c1}{\PYZsh{} cloudsat path latitudes}
        \PY{n}{latc}\PY{o}{.}\PY{n}{detach}\PY{p}{(}\PY{p}{)}
        
        \PY{n}{lonc} \PY{o}{=} \PY{n}{vsc}\PY{o}{.}\PY{n}{attach}\PY{p}{(}\PY{l+s+s1}{\PYZsq{}}\PY{l+s+s1}{Longitude}\PY{l+s+s1}{\PYZsq{}}\PY{p}{)}
        \PY{n}{lonc}\PY{o}{.}\PY{n}{setfields}\PY{p}{(}\PY{l+s+s1}{\PYZsq{}}\PY{l+s+s1}{Longitude}\PY{l+s+s1}{\PYZsq{}}\PY{p}{)}
        \PY{n}{n}\PY{p}{,} \PY{n}{\PYZus{}}\PY{p}{,} \PY{n}{\PYZus{}}\PY{p}{,} \PY{n}{\PYZus{}}\PY{p}{,} \PY{n}{\PYZus{}} \PY{o}{=} \PY{n}{lonc}\PY{o}{.}\PY{n}{inquire}\PY{p}{(}\PY{p}{)}
        \PY{n}{lon\PYZus{}c} \PY{o}{=} \PY{n}{np}\PY{o}{.}\PY{n}{array}\PY{p}{(}\PY{n}{lonc}\PY{o}{.}\PY{n}{read}\PY{p}{(}\PY{n}{n}\PY{p}{)}\PY{p}{)}    \PY{c+c1}{\PYZsh{} cloudsat path longitudes}
        \PY{n}{lonc}\PY{o}{.}\PY{n}{detach}\PY{p}{(}\PY{p}{)}
        \PY{c+c1}{\PYZsh{} converting to [0,360] deg E longitude}
        \PY{n}{lon\PYZus{}c}\PY{p}{[}\PY{n}{lon\PYZus{}c} \PY{o}{\PYZlt{}} \PY{l+m+mi}{0}\PY{p}{]} \PY{o}{=} \PY{n}{lon\PYZus{}c}\PY{p}{[}\PY{n}{lon\PYZus{}c} \PY{o}{\PYZlt{}} \PY{l+m+mi}{0}\PY{p}{]} \PY{o}{+} \PY{l+m+mi}{360}
\end{Verbatim}

    \begin{Verbatim}[commandchars=\\\{\}]
{\color{incolor}In [{\color{incolor}9}]:} \PY{c+c1}{\PYZsh{}c.datasets()}
\end{Verbatim}

    \begin{Verbatim}[commandchars=\\\{\}]
{\color{incolor}In [{\color{incolor}10}]:} \PY{n}{dshgt\PYZus{}c} \PY{o}{=} \PY{n}{c}\PY{o}{.}\PY{n}{select}\PY{p}{(}\PY{l+s+s1}{\PYZsq{}}\PY{l+s+s1}{Height}\PY{l+s+s1}{\PYZsq{}}\PY{p}{)}
         \PY{n}{dsref\PYZus{}c} \PY{o}{=} \PY{n}{c}\PY{o}{.}\PY{n}{select}\PY{p}{(}\PY{l+s+s1}{\PYZsq{}}\PY{l+s+s1}{Radar\PYZus{}Reflectivity}\PY{l+s+s1}{\PYZsq{}}\PY{p}{)}
         
         \PY{n}{hgt\PYZus{}ci} \PY{o}{=} \PY{n}{dshgt\PYZus{}c}\PY{o}{.}\PY{n}{get}\PY{p}{(}\PY{p}{)}
         \PY{n}{ref\PYZus{}ci} \PY{o}{=} \PY{n}{dsref\PYZus{}c}\PY{o}{.}\PY{n}{get}\PY{p}{(}\PY{p}{)}
\end{Verbatim}

    As the data descriptors are a headache and a half to by-hand pull out of
the HDF vgroups, they were inspected manually with HDFVIEW and imported
here. Height were unscaled/unoffset, Reflectivity were scaled by 100.

    \begin{Verbatim}[commandchars=\\\{\}]
{\color{incolor}In [{\color{incolor}11}]:} \PY{n}{hgtc\PYZus{}name} \PY{o}{=} \PY{l+s+s1}{\PYZsq{}}\PY{l+s+s1}{Height of range bin in Reflectivity/Cloud Mask above reference surface (\PYZti{}mean sea level).}\PY{l+s+s1}{\PYZsq{}}
         \PY{n}{hgtc\PYZus{}units} \PY{o}{=} \PY{l+s+s1}{\PYZsq{}}\PY{l+s+s1}{m}\PY{l+s+s1}{\PYZsq{}}
         
         \PY{n}{vra} \PY{o}{=} \PY{p}{[}\PY{o}{\PYZhy{}}\PY{l+m+mi}{5000}\PY{p}{,} \PY{l+m+mi}{30000}\PY{p}{]}
         \PY{n}{\PYZus{}FillValue} \PY{o}{=} \PY{o}{\PYZhy{}}\PY{l+m+mi}{9999}
         
         \PY{n}{invalid} \PY{o}{=} \PY{n}{np}\PY{o}{.}\PY{n}{logical\PYZus{}or}\PY{p}{(}\PY{n}{hgt\PYZus{}ci} \PY{o}{\PYZlt{}} \PY{n}{vra}\PY{p}{[}\PY{l+m+mi}{0}\PY{p}{]}\PY{p}{,} \PY{n}{hgt\PYZus{}ci} \PY{o}{\PYZgt{}} \PY{n}{vra}\PY{p}{[}\PY{l+m+mi}{1}\PY{p}{]}\PY{p}{,} \PY{n}{hgt\PYZus{}ci} \PY{o}{==} \PY{n}{\PYZus{}FillValue}\PY{p}{)}
         \PY{n}{hgt\PYZus{}c} \PY{o}{=} \PY{n}{hgt\PYZus{}ci}\PY{o}{.}\PY{n}{astype}\PY{p}{(}\PY{n+nb}{float}\PY{p}{)}      \PY{c+c1}{\PYZsh{} cleaned height data}
         \PY{n}{hgt\PYZus{}c}\PY{p}{[}\PY{n}{invalid}\PY{p}{]} \PY{o}{=} \PY{n}{np}\PY{o}{.}\PY{n}{nan}
\end{Verbatim}

    \begin{Verbatim}[commandchars=\\\{\}]
{\color{incolor}In [{\color{incolor}12}]:} \PY{n}{refc\PYZus{}name} \PY{o}{=} \PY{l+s+s1}{\PYZsq{}}\PY{l+s+s1}{Radar Reflectivity Factor}\PY{l+s+s1}{\PYZsq{}}
         \PY{n}{refc\PYZus{}units} \PY{o}{=} \PY{l+s+s1}{\PYZsq{}}\PY{l+s+s1}{dBZe}\PY{l+s+s1}{\PYZsq{}}
         \PY{n}{refc\PYZus{}scale} \PY{o}{=} \PY{l+m+mi}{100}
         
         \PY{n}{vra} \PY{o}{=} \PY{p}{[}\PY{o}{\PYZhy{}}\PY{l+m+mi}{4000}\PY{p}{,} \PY{l+m+mi}{5000}\PY{p}{]}
         \PY{n}{\PYZus{}FillValue} \PY{o}{=} \PY{o}{\PYZhy{}}\PY{l+m+mi}{8192}
         
         \PY{n}{invalid} \PY{o}{=} \PY{n}{np}\PY{o}{.}\PY{n}{logical\PYZus{}or}\PY{p}{(}\PY{n}{ref\PYZus{}ci} \PY{o}{\PYZlt{}} \PY{n}{vra}\PY{p}{[}\PY{l+m+mi}{0}\PY{p}{]}\PY{p}{,} \PY{n}{ref\PYZus{}ci} \PY{o}{\PYZgt{}} \PY{n}{vra}\PY{p}{[}\PY{l+m+mi}{1}\PY{p}{]}\PY{p}{,} \PY{n}{ref\PYZus{}ci} \PY{o}{==} \PY{n}{\PYZus{}FillValue}\PY{p}{)}
         \PY{n}{ref\PYZus{}c\PYZus{}pre} \PY{o}{=} \PY{n}{ref\PYZus{}ci}\PY{o}{.}\PY{n}{astype}\PY{p}{(}\PY{n+nb}{float}\PY{p}{)}
         \PY{n}{ref\PYZus{}c\PYZus{}pre}\PY{p}{[}\PY{n}{invalid}\PY{p}{]} \PY{o}{=} \PY{n}{np}\PY{o}{.}\PY{n}{nan}
         \PY{n}{ref\PYZus{}c} \PY{o}{=} \PY{n}{ref\PYZus{}c\PYZus{}pre} \PY{o}{/} \PY{n}{refc\PYZus{}scale}    \PY{c+c1}{\PYZsh{} cleaned and scaled cloudsat reflectivity data}
\end{Verbatim}

    Finally, we'll close all of our open files and datasets.
m.end()
hm.close()

vsc.end()
c.end()
hc.close()
    \subsection{Now we can do some data
analysis!}\label{now-we-can-do-some-data-analysis}

    \paragraph{Subsetting and co-locating the
data}\label{subsetting-and-co-locating-the-data}

So, now we need to narrow down the cloudsat granule to our MODIS domain.

    \begin{Verbatim}[commandchars=\\\{\}]
{\color{incolor}In [{\color{incolor}13}]:} \PY{n}{nlat\PYZus{}c\PYZus{}mask} \PY{o}{=} \PY{n}{np}\PY{o}{.}\PY{n}{logical\PYZus{}and}\PY{p}{(}\PY{n}{lat\PYZus{}c} \PY{o}{\PYZgt{}}\PY{o}{=} \PY{n}{lat\PYZus{}m}\PY{o}{.}\PY{n}{min}\PY{p}{(}\PY{p}{)}\PY{p}{,} \PY{n}{lat\PYZus{}c} \PY{o}{\PYZlt{}}\PY{o}{=} \PY{n}{lat\PYZus{}m}\PY{o}{.}\PY{n}{max}\PY{p}{(}\PY{p}{)}\PY{p}{)}
         \PY{n}{nlon\PYZus{}c\PYZus{}mask} \PY{o}{=} \PY{n}{np}\PY{o}{.}\PY{n}{logical\PYZus{}and}\PY{p}{(}\PY{n}{lon\PYZus{}c} \PY{o}{\PYZgt{}}\PY{o}{=} \PY{n}{lon\PYZus{}m}\PY{o}{.}\PY{n}{min}\PY{p}{(}\PY{p}{)}\PY{p}{,} \PY{n}{lon\PYZus{}c} \PY{o}{\PYZlt{}}\PY{o}{=} \PY{n}{lon\PYZus{}m}\PY{o}{.}\PY{n}{max}\PY{p}{(}\PY{p}{)}\PY{p}{)}
         \PY{n}{nref\PYZus{}c\PYZus{}mask} \PY{o}{=} \PY{n}{np}\PY{o}{.}\PY{n}{logical\PYZus{}and}\PY{p}{(}
             \PY{n}{nlat\PYZus{}c\PYZus{}mask}\PY{o}{.}\PY{n}{flatten}\PY{p}{(}\PY{p}{)} \PY{o}{==} \PY{k+kc}{True}\PY{p}{,} \PY{n}{nlon\PYZus{}c\PYZus{}mask}\PY{o}{.}\PY{n}{flatten}\PY{p}{(}\PY{p}{)} \PY{o}{==} \PY{k+kc}{True}\PY{p}{)}
         
         \PY{n}{nlat\PYZus{}c} \PY{o}{=} \PY{n}{lat\PYZus{}c}\PY{p}{[}\PY{n}{nref\PYZus{}c\PYZus{}mask}\PY{p}{]}
         \PY{n}{nlon\PYZus{}c} \PY{o}{=} \PY{n}{lon\PYZus{}c}\PY{p}{[}\PY{n}{nref\PYZus{}c\PYZus{}mask}\PY{p}{]}
         \PY{n}{nref\PYZus{}c} \PY{o}{=} \PY{n}{ref\PYZus{}c}\PY{p}{[}\PY{n}{nref\PYZus{}c\PYZus{}mask}\PY{p}{]}
         \PY{n}{nhgt\PYZus{}c} \PY{o}{=} \PY{n}{hgt\PYZus{}c}\PY{p}{[}\PY{n}{nref\PYZus{}c\PYZus{}mask}\PY{p}{]}
\end{Verbatim}

    Now that we have Cloudsat narrowed down to at least the rough MODIS
domain, we will attempt to select all of the nearest points to the
Cloudsat orbit track. First, we re-created the initial MODIS plot to
include the subset Cloudsat track overlayed, to make sure our domains
were lining up correctly.

    \begin{Verbatim}[commandchars=\\\{\}]
{\color{incolor}In [{\color{incolor}14}]:} \PY{k+kn}{import} \PY{n+nn}{matplotlib}\PY{n+nn}{.}\PY{n+nn}{pyplot} \PY{k}{as} \PY{n+nn}{plt}
         \PY{k+kn}{import} \PY{n+nn}{cartopy}\PY{n+nn}{.}\PY{n+nn}{crs} \PY{k}{as} \PY{n+nn}{ccrs}
         
         \PY{n}{fig} \PY{o}{=} \PY{n}{plt}\PY{o}{.}\PY{n}{figure}\PY{p}{(}\PY{p}{)}
         \PY{n}{ax} \PY{o}{=} \PY{n}{plt}\PY{o}{.}\PY{n}{axes}\PY{p}{(}\PY{n}{projection}\PY{o}{=}\PY{n}{ccrs}\PY{o}{.}\PY{n}{Mercator}\PY{p}{(}\PY{p}{)}\PY{p}{)}
         \PY{n}{ax}\PY{o}{.}\PY{n}{set\PYZus{}xticks}\PY{p}{(}\PY{n}{np}\PY{o}{.}\PY{n}{arange}\PY{p}{(}\PY{l+m+mi}{163}\PY{p}{,} \PY{l+m+mi}{192}\PY{p}{,} \PY{l+m+mi}{4}\PY{p}{)}\PY{p}{)}
         \PY{n}{ax}\PY{o}{.}\PY{n}{set\PYZus{}yticks}\PY{p}{(}\PY{n}{np}\PY{o}{.}\PY{n}{arange}\PY{p}{(}\PY{o}{\PYZhy{}}\PY{l+m+mi}{4}\PY{p}{,} \PY{l+m+mi}{21}\PY{p}{,} \PY{l+m+mi}{4}\PY{p}{)}\PY{p}{)}
         \PY{n}{ax}\PY{o}{.}\PY{n}{set\PYZus{}xlabel}\PY{p}{(}\PY{l+s+s1}{\PYZsq{}}\PY{l+s+s1}{Longitude}\PY{l+s+s1}{\PYZsq{}}\PY{p}{)}
         \PY{n}{ax}\PY{o}{.}\PY{n}{set\PYZus{}ylabel}\PY{p}{(}\PY{l+s+s1}{\PYZsq{}}\PY{l+s+s1}{Latitude}\PY{l+s+s1}{\PYZsq{}}\PY{p}{)}
         \PY{n}{fig}\PY{o}{.}\PY{n}{suptitle}\PY{p}{(}\PY{n}{ctname\PYZus{}m}\PY{p}{)}
         \PY{n}{c} \PY{o}{=} \PY{n}{plt}\PY{o}{.}\PY{n}{contourf}\PY{p}{(}\PY{n}{lon\PYZus{}m}\PY{p}{,} \PY{n}{lat\PYZus{}m}\PY{p}{,} \PY{n}{ct\PYZus{}m}\PY{p}{,} \PY{n}{cmap}\PY{o}{=}\PY{l+s+s1}{\PYZsq{}}\PY{l+s+s1}{Blues\PYZus{}r}\PY{l+s+s1}{\PYZsq{}}\PY{p}{,} \PY{n}{transform}\PY{o}{=}\PY{n}{ccrs}\PY{o}{.}\PY{n}{Mercator}\PY{p}{(}\PY{p}{)}\PY{p}{)}
         \PY{n}{cb} \PY{o}{=} \PY{n}{plt}\PY{o}{.}\PY{n}{colorbar}\PY{p}{(}\PY{n}{c}\PY{p}{)}
         \PY{n}{cb}\PY{o}{.}\PY{n}{set\PYZus{}label}\PY{p}{(}\PY{l+s+s1}{\PYZsq{}}\PY{l+s+s1}{Meters}\PY{l+s+s1}{\PYZsq{}}\PY{p}{)}
         \PY{n}{ax2} \PY{o}{=} \PY{n}{plt}\PY{o}{.}\PY{n}{axes}\PY{p}{(}\PY{n}{projection}\PY{o}{=}\PY{n}{ccrs}\PY{o}{.}\PY{n}{Mercator}\PY{p}{(}\PY{p}{)}\PY{p}{)}
         \PY{n}{cs} \PY{o}{=} \PY{n}{plt}\PY{o}{.}\PY{n}{plot}\PY{p}{(}\PY{n}{nlon\PYZus{}c}\PY{p}{,} \PY{n}{nlat\PYZus{}c}\PY{p}{,} \PY{l+s+s1}{\PYZsq{}}\PY{l+s+s1}{k\PYZhy{}\PYZhy{}}\PY{l+s+s1}{\PYZsq{}}\PY{p}{)}
         \PY{n}{fig}\PY{o}{.}\PY{n}{savefig}\PY{p}{(}\PY{l+s+s1}{\PYZsq{}}\PY{l+s+s1}{figures/modis\PYZus{}cloud\PYZhy{}top\PYZhy{}height.png}\PY{l+s+s1}{\PYZsq{}}\PY{p}{)}
\end{Verbatim}

    \begin{center}
    \adjustimage{max size={0.9\linewidth}{0.9\paperheight}}{radtran_files/radtran_27_0.png}
    \end{center}
    { \hspace*{\fill} \\}
    
    And we can re-create our initial Cloudsat profile to actually line up
with our MODIS domain.

    \begin{Verbatim}[commandchars=\\\{\}]
{\color{incolor}In [{\color{incolor}15}]:} \PY{n}{fig} \PY{o}{=} \PY{n}{plt}\PY{o}{.}\PY{n}{figure}\PY{p}{(}\PY{p}{)}
         \PY{n}{fig}\PY{o}{.}\PY{n}{suptitle}\PY{p}{(}\PY{n}{refc\PYZus{}name}\PY{p}{)}
         \PY{n}{ax} \PY{o}{=} \PY{n}{plt}\PY{o}{.}\PY{n}{axes}\PY{p}{(}\PY{p}{)}
         \PY{n}{c} \PY{o}{=} \PY{n}{plt}\PY{o}{.}\PY{n}{contourf}\PY{p}{(}\PY{n}{nref\PYZus{}c}\PY{o}{.}\PY{n}{T}\PY{p}{)}
         \PY{n}{ax}\PY{o}{.}\PY{n}{invert\PYZus{}yaxis}\PY{p}{(}\PY{p}{)}
         \PY{n}{ax}\PY{o}{.}\PY{n}{set\PYZus{}ylabel}\PY{p}{(}\PY{l+s+s1}{\PYZsq{}}\PY{l+s+s1}{Range Bin}\PY{l+s+s1}{\PYZsq{}}\PY{p}{)}
         \PY{n}{ax}\PY{o}{.}\PY{n}{set\PYZus{}xlabel}\PY{p}{(}\PY{l+s+s1}{\PYZsq{}}\PY{l+s+s1}{Orbit Bin in MODIS Domain (Southeast to Northwest)}\PY{l+s+s1}{\PYZsq{}}\PY{p}{)}
         \PY{n}{cb} \PY{o}{=} \PY{n}{plt}\PY{o}{.}\PY{n}{colorbar}\PY{p}{(}\PY{p}{)}
         \PY{n}{cb}\PY{o}{.}\PY{n}{set\PYZus{}label}\PY{p}{(}\PY{n}{refc\PYZus{}units}\PY{p}{)}
         \PY{n}{fig}\PY{o}{.}\PY{n}{savefig}\PY{p}{(}\PY{l+s+s1}{\PYZsq{}}\PY{l+s+s1}{figures/cloudsat\PYZus{}radar\PYZhy{}reflectivity.png}\PY{l+s+s1}{\PYZsq{}}\PY{p}{)}
\end{Verbatim}

    \begin{center}
    \adjustimage{max size={0.9\linewidth}{0.9\paperheight}}{radtran_files/radtran_29_0.png}
    \end{center}
    { \hspace*{\fill} \\}
    
    Here we will make our cloud-top height profile from the cloudsat
reflectivity by selecting the maximum height of the highest reflectivity
bin greater than or equal to -18 dBZe (Wang et al. 2017).

    \begin{Verbatim}[commandchars=\\\{\}]
{\color{incolor}In [{\color{incolor}38}]:} \PY{n}{profile\PYZus{}cloudsat} \PY{o}{=} \PY{n}{np}\PY{o}{.}\PY{n}{empty}\PY{p}{(}\PY{n+nb}{len}\PY{p}{(}\PY{n}{nref\PYZus{}c}\PY{p}{[}\PY{p}{:}\PY{p}{,}\PY{l+m+mi}{0}\PY{p}{]}\PY{p}{)}\PY{p}{)}
         
         \PY{k}{for} \PY{n}{s} \PY{o+ow}{in} \PY{n+nb}{range}\PY{p}{(}\PY{n+nb}{len}\PY{p}{(}\PY{n}{nref\PYZus{}c}\PY{p}{[}\PY{p}{:}\PY{p}{,}\PY{l+m+mi}{0}\PY{p}{]}\PY{p}{)}\PY{p}{)}\PY{p}{:}
             \PY{k}{for} \PY{n}{z} \PY{o+ow}{in} \PY{n+nb}{range}\PY{p}{(}\PY{n+nb}{len}\PY{p}{(}\PY{n}{nref\PYZus{}c}\PY{p}{[}\PY{l+m+mi}{0}\PY{p}{,}\PY{p}{:}\PY{p}{]}\PY{p}{)}\PY{p}{)}\PY{p}{:}
                 \PY{k}{if} \PY{n}{nref\PYZus{}c}\PY{p}{[}\PY{n}{s}\PY{p}{,}\PY{n}{z}\PY{p}{]} \PY{o}{\PYZgt{}}\PY{o}{=} \PY{o}{\PYZhy{}}\PY{l+m+mi}{18}\PY{p}{:}
                     \PY{n}{profile\PYZus{}cloudsat}\PY{p}{[}\PY{n}{s}\PY{p}{]} \PY{o}{=} \PY{n}{nhgt\PYZus{}c}\PY{p}{[}\PY{n}{s}\PY{p}{,}\PY{n}{z}\PY{p}{]}
                     \PY{k}{break}
                 \PY{k}{else}\PY{p}{:}
                     \PY{k}{continue}
         
         \PY{n}{profile\PYZus{}cloudsat}\PY{p}{[}\PY{n}{profile\PYZus{}cloudsat} \PY{o}{\PYZlt{}} \PY{l+m+mi}{0}\PY{p}{]} \PY{o}{=} \PY{l+m+mi}{0}
         
         \PY{n}{fig} \PY{o}{=} \PY{n}{plt}\PY{o}{.}\PY{n}{figure}\PY{p}{(}\PY{p}{)}
         \PY{n}{fig}\PY{o}{.}\PY{n}{suptitle}\PY{p}{(}\PY{l+s+s1}{\PYZsq{}}\PY{l+s+s1}{Height Profile of Cloudsat Reflectivity \PYZgt{} \PYZhy{}18 dBZe}\PY{l+s+s1}{\PYZsq{}}\PY{p}{)}
         \PY{n}{ax} \PY{o}{=} \PY{n}{plt}\PY{o}{.}\PY{n}{axes}\PY{p}{(}\PY{p}{)}
         \PY{n}{p} \PY{o}{=} \PY{n}{plt}\PY{o}{.}\PY{n}{plot}\PY{p}{(}\PY{n}{profile\PYZus{}cloudsat}\PY{p}{)}
         \PY{n}{ax}\PY{o}{.}\PY{n}{set\PYZus{}ylabel}\PY{p}{(}\PY{l+s+s1}{\PYZsq{}}\PY{l+s+s1}{Height [m]}\PY{l+s+s1}{\PYZsq{}}\PY{p}{)}
         \PY{n}{ax}\PY{o}{.}\PY{n}{set\PYZus{}xlabel}\PY{p}{(}\PY{l+s+s1}{\PYZsq{}}\PY{l+s+s1}{Orbit Bin in MODIS Domain (Southeast to Northwest)}\PY{l+s+s1}{\PYZsq{}}\PY{p}{)}
         \PY{n}{fig}\PY{o}{.}\PY{n}{savefig}\PY{p}{(}\PY{l+s+s1}{\PYZsq{}}\PY{l+s+s1}{figures/cloudsat\PYZus{}cloudtop\PYZhy{}height\PYZhy{}profile.png}\PY{l+s+s1}{\PYZsq{}}\PY{p}{)}
\end{Verbatim}

    \begin{center}
    \adjustimage{max size={0.9\linewidth}{0.9\paperheight}}{radtran_files/radtran_31_0.png}
    \end{center}
    { \hspace*{\fill} \\}
    
    Here marks where we end our analysis journey, as we struggle to deal
with the data structures problem necessary. We attempted loops, distance
minimization, and more where the small details and headaches were never
isolated and worked out.

    \section{Conclusion}\label{conclusion}

It is difficult to draw a meaningful conclusion from this project, as we
never quite reached the quantitative goals we set out to achieve. Due to
data structures and programming related headaches, we were not able to
produce the MODIS profile for direct comparison to the Cloudsat profile
we produced. We did take away knowledge regarding the methods employed
to produce these data and the general differences in passive and active
techniques.

Planned and future work would involve co-locating the two datasets to
produce a similar profile to the Cloudsat profile produced above. Then,
simple x-y linear correlation would be employed to investigate biases
between the two data sets. These same techniques could be applied across
many granules in space and time to uncover possible consistent biases
between these two instruments.

    \section{References}\label{references}

Davies, R., and M. Molloy, 2012: Global cloud height fluctuations
measured by MISR on Terra from 2000 to 2010.~Geophysical Research
Letters,~39, L03701, doi:10.1029/2011GL050506.

King, M. D., Y. J. Kaufman, W. P. Menzel, and D. Tanre, 1992: Remote
sensing of cloud, aerosol, and water vapor properties from the Moderate
Resolution Imaging Spectrometer (MODIS). IEEE Trans. Geosci. Remote
Sens., 30, 2-27.

Mahesh, A.,~M. A. Gray,~S. P. Palm,~W. D. Hart, and~J. D.
Spinhirne,~2004:~Passive and active detection of clouds: Comparisons
between MODIS and GLAS observations,~Geophys. Res. Lett.,~31, L04108,
doi:10.1029/2003GL018859.

Marchand, R., G.G. Mace, T. Ackerman, and G. Stephens, 2008: Hydrometeor
Detection Using CloudSat -- An Earth-Orbiting 94-GHz Cloud Radar. J.
Atmos. Oceanic Technol., 25, 519-533.

Wang, Y., Y. Chen, Y. Fu, G. Liu, 2017: Identification of precipitation
onset based on Cloudsat observations. J. Quan. Spec. and Rad. Trans.,
188, 142-147.

Weisz, E.,~J. Li,~W. P. Menzel,~A. K. Heidinger,~B. H. Kahn, and~C.-Y.
Liu,~2007:~Comparison of AIRS, MODIS, CloudSat and CALIPSO cloud top
height retrievals.~Geophys. Res. Lett.,~34, L17811,
doi:10.1029/2007GL030676.

Xu, K.-M. et al, 2007: Statistical analyses of satellite cloud object
data from CERS, part II: Tropical convective cloud objects during the
1998 El Nino and evidence for supporting the fixed anvil temperature
hypothesis. J. Climate, 20, 819-842.


    % Add a bibliography block to the postdoc
    
    
    
    \end{document}
